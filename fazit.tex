% !TeX encoding = UTF-8
\section{Fazit}
Die Antwort auf die Frage, wann und wozu die pädagogische Praxis einer Traumap{\"a}dagogik bedürfe, ist nicht einfach zu beantworten. Die durch die Vertreter der Traumap{\"a}dagogik genannten Definitionen (siehe Kapitel 1.2) geben keine klare Antwort auf die Frage, was diese sei oder werden wolle (ein traumasensibles Organisationsentwicklungkonzept, eine „Fachdisziplin“ oder Haltung?). Was ihr Gegenstand sei und wer die AdressatInnen seien, kann noch nicht eindeutig beantwortet werden.

\begin{quote}
\small{„Die meisten traumap{\"a}dagogischen Konzepte beziehen sich explizit auf traumatisierte M{\"a}dchen und Jungen. Andere betonen den Nutzen der Traumap{\"a}dagogik f{\"u}r alle Kinder und Jugendlichen. M{\"o}glicherweise werden die traumatisierten M{\"a}dchen und Jungen durch eine ‚Sonder-‘P{\"a}dagogik isoliert und stigmatisiert. Sie brauchen {\"u}berall Konzepte und Strukturen, die traumap{\"a}dagogische Inhalte und Methoden ber{\"u}cksichtigen und erm{\"o}glichen“ (Weiß u.a. 2016: 28).}
\end{quote}

Die Argumente, die die Notwendigkeit der Traumap{\"a}dagogik aufzeigen sollen (siehe Kapitel 3), verdeutlichen deutlich, dass Traumafolgen den Alltag der Kinder und Erwachsenen beeinflussen. Die pädagogischen Interventionen, die die Schwierigkeiten der Betroffenen vor dem Hintergrund der Biografie als verständliche Reaktionen berücksichtigen, können helfen, den Alltag anders zu gestalten. Allerdings sind das Interventionen (vgl. Abbildung 5), die ohne den semantischen Zugriff auf Trauma als pädagogische beschreibbar und praktizierbar sind. Erkennbar ist die unklare Abgrenzung der traumapädagogischen Angebote gegenüber den (psycho-)therapeutischen (vgl. Zimmerman 2016: 47). Der Begriff Traumapädagogik selbst wirft Fragen auf (siehe 4.2).

Trotz dieser eher kritischen Bestandsaufnahme ist es wichtig zu betonen, dass Traumapädagogik wichtige Akzente setzt und einigen Nutzen hat. Vor allem hat traumasensible Pädagogik das Potenzial, präventiv zu wirken. Wird Traumapädagogik als ein Ansatz verstanden, der über Ursachen und Folgen extremer Stresssituationen aufklärt, kann mit Schulungen oder Weiterbildungen dazu beigetragen werden, dass einigen Menschen früher und kompetenter geholfen werden kann. Ebenso fühlen sich PädagogInnen fähiger, wenn sie eine Erklärung für das Verhalten der Kinder und Jugendlichen haben. Die Haltung, die Traumap{\"a}dagogik gegenüber ihren AdressatInnen und MitarbeiterInnen der Einrichtungen fordert, sowie die Berücksichtigung der institutionellen Aspekte ist eine solche, die erstrebenswert für viele, wenn nicht alle pädagogischen Handlungsfelder ist. Eine Pädagogik, die auf individuelle Problemlagen der AdressatInnen eingeht, transparente und tragende Strukturen fordert, die sowohl Kinder und Jugendliche als auch die Erwachsene, partizipativ in die Prozesse einbindet, geschützte, gewaltfreie Räume zur Verfügung stellt, soll jedoch keine solche sein, die nur in Verbindung mit einer Diagnose zugestanden wird. Die Kopplung dieser an einen Begriff, der sehr vielfältig, uneindeutig und stark wirkend ist, ist reflexionswürdig (siehe 4.2). „Trauma ist [...] als individueller und sozialer Prozess eine Realität und gleichzeitig als wissenschaftliches Konstrukt eine Erfindung“ (Becker 2006: 165). Wie auch immer ein Trauma definiert wird, welche Konzepte dominieren und anerkannt werden: Die daraus abgeleiteten Behandlungsmethoden beeinflussen die betroffenen Menschen und deren Chancen auf angemessene Unterstützung.

\begin{quote}
\small{„Trauma kann schon durch die Definition zum Stigma werden, und die sozialwissenschaftliche Entwicklung des Traumadiskurses hat immer sozialpolitische Bedeutung. Traumata können als Ausgrenzung, Manipulation, Auszeichnung, Selbstrechtfertigung etc. benutzt werden“ (Becker 2006: 165).}
\end{quote}

Leider läuft Traumap{\"a}dagogik anscheinend Gefahr, sich auf der Suche nach ihrem Platz zwischen Anpassung an die gegebenen Strukturen und dem Nutzen ihres emanzipatorischen Potenzials für Ersteres zu entscheiden. Traumapädagogik in ihrem Bestreben, psychotraumatologische Erkenntnisse zu propagieren (vgl. BAG-TP 2011: 2), meint den pädagogischen Fachkräften in gebündelter und aufgearbeiteter Form Wissen, Konzepte und Methoden zur Verfügung zu stellen, die ihnen helfen sollen, mit den Anforderungen des beruflichen Alltags besser zurechtzukommen. Dabei wird zugleich eine ganze Reihe von Beschäftigungen und neuen Arbeitsfeldern generiert. TraumaberaterInnen, TraumapädagogInnen, Fort- und Weiterbildungen, Qualifizierungen, Zertifikate, Standards, Arbeitsgruppen etc.: Viel Aufwand wurde betrieben. Ob es dann im Interesse derer, die daran beteiligt sind, ist, dass die Traumap{\"a}dagogik zu einer guten, emanzipierten Pädagogik wird? Bleibt Traumapädagogik bei einem medizinischen, pathologischen Störungsbild der Traumata und des traumatisierten Kindes, so wird sie zu einer exklusiven Pädagogik, die Haltungen, Methoden und Strukturen fordert, die gut für alle sind, aber aufgrund der Benennung der besonders bedürftigen Zielgruppe und der knappen Ressourcen nur wenigen zugestanden werden. Damit wird die strukturelle Versorgungsungleichheit reproduziert. Soll Traumapädagogik allen Kinder und Jugendlichen zur Verfügung stehen, muss sie entweder ihr Traumaverständnis sehr breit aufstellen und alles, was dem Menschen widerfährt und ihn überfordert oder als massiver Stress empfunden wird, als Trauma definieren oder anders auf den Traumabegriff verzichten. Worüber würden wir aber sprechen, wenn wir den Traumabegriff nicht hätten? (vgl. K{\"u}hner 2005). Was wäre Traumap{\"a}dagogik ohne Trauma?
