% !TeX encoding = UTF-8
\section{Literaturverzeichnis}

\setlength{\parindent}{0pt}
\hang
Anhorn, R. \& Balzereit, M. (2016). Die »Arbeit am Sozialen« als »Arbeit am Selbst« - Herrschaft, Soziale Arbeit und die therapeutische Regierungsweise im Neo-Liberalismus: Einf{\"u}hrende Skizzierung eines Theorie- und Forschungsprogramms. In R. Anhorn \& M. Balzereit (Hrsg.), \textit{Handbuch Therapeutisierung und Soziale Arbeit} (S. 3-207). Wiesbaden: Springer.

\hang
Bausum, J. (2013). Ressourcen der Gruppe zur Selbstbem{\"a}chtigung. „Ich bin und ich brauche euch". In J. Bausum, L. U. Bessr, M. Kühn \& W. Weiß (Hrsg.), \textit{Traumapädagogik: Grundlagen}, \textit{Arbeitsfelder und Methoden für die pädagogische Praxis} (S. 179-188). Weinheim: Beltz.

\hang
Bausum, J., Bessr, L. U., Kühn, M. \& Weiß, W. (Hrsg.). (2013). \textit{Traumapädagogik: Grundlagen, Arbeitsfelder und Methoden für die pädagogische Praxis.} Weinheim: Beltz.

\hang
Becker, D. (2006). \textit{Die Erfindung des Traumas: verflochtene Geschichten}. Gießen: Psychosozial.

\hang
Bundesarbeitsgemeinschaft Traumap{\"a}dagogik – BAG-TP (2011). \textit{Standards für traumapädagogische Konzepte in der stationären Kinder- und Jugendhilfe.} Gnarrenburg: BAG-Traumapädagogik. Verfügbar unter: http://www.bag-traumapaedagogik.de/index.php/standards.html\break[22.03.2017].

\hang
Bundesministerium für Familie, Senioren, Frauen und Jugend – BMFSFJ (2009). \textit{13. Kinder- und Jugendbericht Bericht über die Lebenssituation junger Menschen und die Leistungen der Kinder- und Jugendhilfe.} Berlin. Verfügbar unter:\\ https://www.bmfsfj.de/blob/93144/f5f2144cfc504efbc6574af8a1f30455/13-kinder-jugend-bericht-data.pdf [25.04.2017].

\hang
Ding, U. (2013). Trauma und Schule. Was l{\"a}sst Peter wieder lernen? {\"U}ber unsichere Bedingungen und sichere Orte in der Schule. In J. Bausum, L. U. Bessr, M. Kühn \& W. Weiß (Hrsg.), \textit{Traumapädagogik: Grundlagen, Arbeitsfelder und Methoden für die pädagogische Praxis} (S. 55-66). Weinheim: Beltz.

\hang
Fachverband Traumapädagogik - DeGPT (o. J.). \textit{Anerkannte Ausbildungsinstitute für Traumapädagogik und Traumazentrierte Fachberatung.} Verfügbar unter: http://www.degpt.de/DeGPT-Dateien/Institute-Traumap\%C3\%A4dagogik-April\%202017.pdf [08.05.2017].

\hang
Falkai, P., Wittchen, H.-U. \& American Psychiatric Association (APA). (2015). \textit{Diagnostisches und statistisches Manual psychischer Störungen DSM-5 / American Psychiatric Association; Deutsche Ausgabe.} G{\"o}ttingen: Hogrefe.

\hang
Fegert, J. M., Ziegenhain, U. \& Goldbeck, L. (Hrsg.). (2013). \textit{Traumatisierte Kinder und Jugendliche in Deutschland Analysen und Empfehlungen zu Versorgung und Betreuung.} Weinheim: Beltz.

\hang
Freud, S. (1961). \textit{Gesammelte Werke. 11, Vorlesungen zur Einführung in die Psychoanalyse.} Fischer Verlag: Frankfurt am Main.

\hang
Gahleitner, S. B. (2011). \textit{Das Therapeutische Milieu in der Arbeit mit Kindern und Jugendlichen: Trauma- und Beziehungsarbeit in stationären Einrichtungen.} Bonn: Psychiatrie-Verlag.

\hang
Gahleitner, S. B. \& Homfeld, H. G. (2016). Kooperation und psychosoziale Traumaarbeit. In W. Weiß, T. Kessler \& S. B. Gahleitner (Hrsg.), \textit{Handbuch Traumapädagogik} (S. 320-326). Weinheim: Beltz.

\hang
Hantke, L. (2012). Traumazentrierte Arbeit im psychosozialen Feld. Unterschiede und Gemeinsamkeiten von Traumatherapie, -beratung und -pädagogik. In \textit{Trauma \& Gewalt, 3} (6), 198-205. Stuttgart: Klett-Cotta.

\hang
Hantke, L. (2015). Traumakompetenz in psychosozialen Handlungsfeldern. In S. B. Gahleitner, C. Frank \& A. Leitner (Hrsg.), \textit{Ein Trauma ist mehr als ein Trauma : biopsychosoziale Traumakonzepte in Psychotherapie, Beratung, Supervision und Traumapädagogik} (S. 118-126). Weinheim: Beltz.

\hang
Hantke, L. \& Görges, H. (2012). \textit{Handbuch Traumakompetenz: Basiswissen für Therapie, Beratung und Pädagogik.} Paderborn: Junfermann.

\hang
Hausmann, C. (2006). \textit{Einführung in die Psychotraumatologie.} Wien: Facultas.

\hang
Hensel, T. (2013). Traumatherapie bei Kindern und Jugendlichen. Ausbildungs- und Versorgungsrealit{\"a}t aus der Sicht eines niedergelassenen Kinder- und Jugendlichenpsychotherapeuten und Ausbilders in Kindertraumapsychotherapie. In J. M. Fegert, U. Ziegenhain \& L. Goldbeck (Hrsg.), \textit{Traumatisierte Kinder und Jugendliche in Deutschland.} Analysen und Empfehlungen zu Versorgung und Betreuung (S. 82-88).Weinheim: Beltz.

\hang
Hensel, T. (2014). Die Psychotraumatologie des Kindes- und Jugendalters. In S. Gahleitner, T. Hensel, M. Baierl, M. K{\"u}hn \& M. Schmid (Hrsg.), \textit{Traumap{\"a}dagogik in psychosozialen Handlungsfeldern. Ein Handbuch f{\"u}r Jugendhilfe, Schule und Klinik} (S. 27-40). Göttingen: Vandenhoeck \& Ruprecht.

\hang
Huber, M. (2003). \textit{Trauma und Traumabehandlung. 1. Trauma und die Folgen.} Paderborn: Junfermann.

\hang
ICD-10-GM. (2016). \textit{Die Internationale statistische Klassifikation der Krankheiten und verwandter Gesundheitsprobleme, 10. Revision, German Modification.} Verfügbar unter:\\ https://www.dimdi.de/static/de/klassi/icd-10-gm/kodesuche/onlinefassungen\\
/htmlgm2016/block-f40-f48.htm [23.01.2017].

\hang
Kessler, T. (2016). {\"A}ußere Eindr{\"u}cke und innere Erwartungen. Theoretische Aspekte zu den Dynamiken von {\"U}bertragung und Gegenreaktion in der traumap{\"a}dagogischen Arbeit. In W. Weiß, T. Kessler \& S. B. Gahleitner (Hrsg.), \textit{Handbuch Traumapädagogik} (S. 123-130). Weinheim: Beltz.

\hang
Kühn, M. (2006). \textit{Bausteine einer „P{\"a}dagogik des Sicheren Ortes“ - Aspekte eines p{\"a}dagogischen Umgangs mit (traumatisierten) Kindern in der Jugendhilfe aus der Praxis des SOS-Kinderdorfes Worpswede.} Vortrag: Fachtagung „(Akut) traumatisierte Kinder und Jugendliche in P{\"a}dagogik und Jugendhilfe“. Merseburg, 17./18.02.2006. Verfügbar unter:\\ http://www.jugendsozialarbeit.de/media/raw/martin\_kuehn.pdf [24.02.2017].

\hang
Kühn, M. (2013). Macht eure Welt endlich wieder mit zu meiner. Anmerkungen zum Begriff der Traumapädagogik. In J. Bausum, L. U. Besser, M. Kühn \& W. Weiß (Hrsg.), \textit{Traumapädagogik: Grundlagen, Arbeitsfelder und Methoden für die pädagogische Praxis} (S. 24-37). Weinheim: Beltz.

\hang
Kühn, M. (2014). Traumap{\"a}dagogik - von einer Graswurzelbewegung zur Fachdisziplin. In S. Gahleitner, T. Hensel, M. Baierl, M. K{\"u}hn \& M. Schmid (Hrsg.), \textit{Traumap{\"a}dagogik in psychosozialen Handlungsfeldern. Ein Handbuch f{\"u}r Jugendhilfe, Schule und Klinik} (S. 21-26). Göttingen: Vandenhoeck \& Ruprecht.

\hang
K{\"u}hner, A. (2005). Schmerzfreier {\"u}ber Leiden sprechen? Gedankenexperimente und Beobachtungen zu Ambivalenzen des Traumabegriffs. In A. Karger \& R. Heinz (Hrsg.), \textit{Trauma und Schmerz. Psychoanalytische, philosophische und sozialwissenschaftliche Perspektiven} (S. 165-172). Gießen: Psychosozial.

\hang
Landolt, M. \& Hensel, T. (Hrsg.). (2008). \textit{Traumatherapie bei Kindern und Jugendlichen.} Göttingen: Hogrefe.

\hang
Lang, B. (2013). Stabilisierung und (Selbst-) F{\"u}rsorge f{\"u}r p{\"a}dagogische Fachkr{\"a}fte als institutioneller Auftrag. In J. Bausum, L. U. Besser, M. Kühn \& W. Weiß (Hrsg.), \textit{Traumapädagogik: Grundlagen, Arbeitsfelder und Methoden für die pädagogische Praxis} (S. 211-220). Weinheim: Beltz.

\hang
Maercker A. (Hrsg.). (2009). \textit{Posttraumatische Belastungsst{\"o}rungen.} Heidelberg: Springer.

\hang
M{\"o}hrlein, G. \& Hoffart, E.-M. (2014). Traumap{\"a}dagogische Konzepte in der Schule. In S. B. Gahleitner, T. Hensel, M. Baierl, M. K{\"u}hn \& M. Schmid (Hrsg.), \textit{Traumap{\"a}dagogik in psychosozialen Handlungsfeldern. Ein Handbuch f{\"u}r Jugendhilfe, Schule und Klinik} (S. 91-103). Göttingen: Vandenhoeck \& Ruprecht.

\hang
Purtscher-Penz, K. (2015). Traumatisierung in der Kindheit und Jugend. Hilfe durch Psychotherapie und Traumap{\"a}dagogik. In S. B. Gahleitner, Ch. Frank, \& A. Leitner (Hrsg.), \textit{Ein Trauma ist mehr als ein Trauma : biopsychosoziale Traumakonzepte in Psychotherapie, Beratung, Supervision und Traumap{\"a}dagogik} (S. 95-105). Weinheim: Beltz.

\hang
Reinelt, T., Vasileva, M. \& Petermann, F. (2016). Psychische Auff{\"a}lligkeiten von Fl{\"u}chtlingskindern: Eine Blickverengung durch die Posttraumatische Belastungsst{\"o}rung? \textit{Kindheit und Entwicklung}, 25(4), 231-237.

\hang
Rothdeutsch-Granzer, Ch., Weiß, W. \& Gahleitner, S. B. (2015). Traumap{\"a}dagogik – eine junge Fachrichtung mit traditionsreichen Wurzeln und hoffnungsvollen Perspektiven. In S. B. Gahleitner, C. Frank \& A. Leitner (Hrsg.), \textit{Ein Trauma ist mehr als ein Trauma: biopsychosoziale Traumakonzepte in Psychotherapie, Beratung, Supervision und Traumap{\"a}dagogik} (S. 171-186). Weinheim: Beltz.

\hang
Saß, H., Wittchen, H. U., Zaudig, M. \& Houben, I. (2003). \textit{Diagnostisches und Statistisches Manual Psychischer Störungen – Textrevision – DSM-IV-TR.} Göttingen: Hogrefe.

\hang
Schirmer, C. (2016). Die Entwicklung der traumapädagogischen Standards. Ein Meilenstein in der stationären Erziehungshilfe. In W. Weiß, T. Kessler \& S. B. Gahleitner (Hrsg.), \textit{Handbuch Traumapädagogik} (S.439-448). Weinheim: Beltz.

\hang
Schmid, M. (2013). Umgang mit traumatisierten Kindern und Jugendlichen in der station{\"a}ren Jugendhilfe: „Traumasensibilit{\"a}t“ und „Traumap{\"a}dagogik“. In J. M. Fegert, U. Ziegenhain \& L. Goldbeck (Hrsg.), \textit{Traumatisierte Kinder und Jugendliche in Deutschland Analysen und Empfehlungen zu Versorgung und Betreuung} (S. 36-60). Weinheim: Beltz.

\hang
Schmid, M. \& Lang, B. (2012). Was ist das Innovative und Neue an einer Traumap{\"a}dagogik? In M. Schmid, M. Tetzer, K. Rensch \& S. Schlüter- Müller (Hrsg.), \textit{Handbuch Psychiatriebezogene Sozialpädagogik} (S. 337-351). Göttingen: Vandenhoeck \& Ruprecht.

\hang
Schmid, M., Fegert, J. M. \& Petermann, F. (2010). Traumaentwicklungsstörung: Pro und Contra. \textit{Kindheit und Entwicklung, 19} (1), 47-63.

\hang
Schmid, M., Purtscher-Penz, K., Stellermann-Strehlow, K. (2014). Traumasensibilit{\"a}t und traumap{\"a}dagogische Konzepte in der Kinder- und Jugendpsychiatrie/-psychotherapie. In S. B. Gahleitner, T. Hensel, M. Baierl, M. K{\"u}hn \& M. Schmid (Hrsg.), \textit{Traumap{\"a}dagogik in psychosozialen Handlungsfeldern Ein Handbuch f{\"u}r Jugendhilfe, Schule und Klinik} (S. 174-191). Göttingen: Vandenhoeck \& Ruprecht.

\hang
Schmid, M.,Wiesinger, D., Lang, B., Jaszkowic, K. \& Fegert, J. M. (2007). Brauchen wir eine Traumap{\"a}dagogik? – Ein Pl{\"a}doyer f{\"u}r die Entwicklung und Evaluation von traumap{\"a}dagogischen Handlungskonzepten in der station{\"a}ren Jugendhilfe. \textit{KONTEXT} 38(4), 330-357.

\hang
Strauß, J.W. (2016). Grenzen der Traumap{\"a}dagogik – kritische (Nach-)Fragen. In W. Weiß, T. Kessler \& S. B. Gahleitner (Hrsg.), \textit{Handbuch Traumapädagogik} (S. 449-457). Weinheim: Beltz.

\hang
Uttendörfer, J. (2008). Traumazentrierte Pädagogik. Von der Entwicklung der Kultur eines „Sicheren Ortes“. \textit{Unsere Jugend}, 60(2), 50-6.

\hang
Website Traumapädagogik (o. J.). Herzlich Willkommen. Verfügbar unter:\\ http://www.traumapaedagogik.de/[08.05.2017].

\hang
Weiß, W. (2003). \textit{Philipp sucht sein Ich: Zum pädagogischen Umgang mit Traumata in den Erziehungshilfen} (1. Aufl.). Weinheim: Votum.

\hang
Weiß, W. (2013). \textit{Philipp sucht sein Ich: Zum pädagogischen Umgang mit Traumata in den Erziehungshilfen} (7. Aufl.). Weinheim: Juventa.

\hang
Weiß, W. (2016a). Die Pädagogik der Selbstbemächtigung. Eine traumapädagogische Methode. In W. Weiß, T. Kessler \& S. B. Gahleitner (Hrsg.), \textit{Handbuch Traumapädagogik} (S. 290-302). Weinheim: Beltz.

\hang
Weiß, W. (2016b). Traumap{\"a}dagogik: Entstehung, Inspirationen, Konzepte. In W. Weiß, T. Kessler \& S. B. Gahleitner (Hrsg.), \textit{Handbuch Traumapädagogik} (S. 20-32). Weinheim: Beltz.

\hang
Weiß, W., Kessler, T. \& Gahleitner, S. B. (Hrsg.). (2016). \textit{Handbuch Traumapädagogik.} Weinheim: Beltz.

\hang
Zimmermann, D. (2016). \textit{Traumapädagogik in der Schule: Pädagogische Beziehungen mit schwer belasteten Kindern und Jugendlichen.} Gießen: Psychosozial-Verlag.
