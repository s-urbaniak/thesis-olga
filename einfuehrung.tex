% !TeX encoding = UTF-8
\section{Einführung}

Traumapädagogik bezeichnet eine junge Bewegung, die mit lebensgeschichtlich belasteten Kindern und Jugendlichen arbeitende Fachkräfte unterstützt (vgl. Weiß u.a., 2016: 11). Sie ist als Antwort auf die Überforderung der pädagogischen Praxis mit traumatisierten Kindern und Jugendlichen entstanden (vgl. Kühn 2014: 25). In den letzten Jahren entwickelte sich Traumapädagogik von vereinzelten Konzepten hin zu einer institutionalisierten Form mit eigenen Gremien, Zertifizierungen und Standards (vgl. Schimmer 2016: 439 ff.). Entstanden in der stationären Jugendhilfe, öffnet sich Traumapädagogik für andere pädagogische Arbeitsfelder, etwa Primarerziehung, Bildung und Behindertenhilfe als Bereiche, für die psychotraumatologisches Wissen bedeutsam ist (vgl. Bausum u.a. 2013: 8).

Trauma ist ein aus der Medizin stammender Begriff, der erst in den 1960er-Jahren seine psychiatrisch-psychologische Bedeutung einer psychischen Verletzung erhalten hat (vgl. Anhorn \& Balzereit 2016: 37). Trauma für sich ist keine Diagnose im Sinne einer psychischen Störung (vgl. ebd.), ebenso kein pädagogischer Begriff (vgl. Zimmermann 2016: 47). Dennoch ist sie zu einem so attraktiven Begriff für die pädagogische Praxis geworden, dass sich aus dieser Verbindung eine „eigenständige Fachrichtung“ (vgl. z.B. Bausum u.a. 2013:8) entwickelt.

Die vorliegende Arbeit beschäftigt sich mit dem Phänomen der Traumapädagogik und geht der Frage nach, ob die pädagogische Praxis einer Traumapädagogik bedürfe. Wann und wozu und wann nicht mehr? Macht der Zugriff auf \textit{Trauma} die pädagogische Praxis handlungsfähiger?

Um sich den Antworten auf diese Fragen zu nähern, wird zunächst erklärt, was Traumapädagogik ist. Eine eindeutige Definition für diese noch junge und heterogene Bewegung gibt es (noch) nicht. Da \textit{Trauma} semantisch und inhaltlich zur Traumapädagogik gehört, die sich wiederum explizit auf die Psychotraumatologie bezieht, wird im ersten Kapitel der Traumabegriff erklärt. Auf dieser Grundlage werden die Entstehung und Institutionalisierung der „neuen Fachdisziplin“ nachverfolgt. Sind diese Hintergründe geklärt, wird im zweiten Kapitel der Blick frei, um sich mit der in der Traumapädagogik geforderten Haltung, den Inhalten der Konzepte sowie mit Traumapädagogik als institutionellem Konzept auseinanderzusetzen. Der Frage, warum es einer Traumapädagogik bedürfe, widmet sich das dritte Kapitel. Dabei sollen die Argumente für die Traumapädagogik, die in der Literatur zu finden sind, gesammelt werden. Das vierte Kapitel reflektiert kritisch das Vorangegangene und untersucht, wofür die Traumapädagogik benutzt wird. Dabei wird sichtbar, dass Traumapädagogik neue Probleme und Fragen generiert. Anschließend wird der Begriff \textit{Traumapädagogik} kritisch hinterfragt, um noch einmal die Frage nach dem Nutzen des \textit{Traumas} für die pädagogische Praxis anders zu stellen.
