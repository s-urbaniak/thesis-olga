% !TeX encoding = UTF-8
\section{Macht \textit{Trauma} die Praxis handlungsfähiger?}
Die besonderen Bedarfe der traumatisierten Kinder und Jugendlichen und die Überforderung der Fachkräfte sollen die Notwendigkeit der Veränderungen in pädagogischen Institutionen begründen (siehe Kapitel 3). Allerdings wird diesen Bedarfen mit bekannten, pädagogischen und organisatorischen Maßnahmen begegnet. Die kindzentrierten Anteile sind als „sehr konsequente Umsetzung einer guten altbekannten Pädagogik“ (Schmid \& Lang 2012: 343) beschreibbar. Das Neue an der Traumapädagogik besteht darin, die Begründung für die Handlungskonzepte aus der psychotraumatologischen Perspektive herzuleiten (vgl. ebd.). Dabei ist die Verbindung zum Traumabegriff weder eindeutig, noch scheint sie notwendig zu sein. Im Folgenden wird Traumapädagogik auf die Frage hin reflektiert, wem diese einen Nutzen bringe und wozu sie das \textit{Trauma} brauche (4.1). Anschließend wird der Begriff \textit{Traumapädagogik} kritisch hinterfragt (4.2).

\subsection{Gute Gründe für Traumapädagogik? Kritische Gedanken}
Traumapädagogische Leitgedanken bewegen sich im Spannungsfeld zwischen den von den praktisch arbeitenden Fachkräften geforderten Haltungen, ihrem Wissen und ihren anzuwendenden Methoden einerseits und der Adressierung der institutionellen Rahmenbedingungen andererseits, die wiederum nicht losgelöst von den sozialen und politischen existieren. Das Argument, das die Handlungsunfähigkeit der pädagogischen Fachkräfte mit den besonders herausfordernden traumatisierten KlientInnen begründet, kann wirken, als ob die pädagogische Praxis, vor allem in den stationären Jugendhilfe, die schwer belasteten Kinder und Jugendlichen „bräuchte“, um auf ihre sie an sich überfordernden Arbeitsbedingungen aufmerksam zu machen und institutionelle Veränderungen zu legitimieren. Sind das wirklich die traumatisierten, belasteten Kinder und Jugendlichen, die die pädagogische Praxis herausfordern? Die in Kapitel 3 gesammelten Argumente weisen darauf hin, dass betroffene Kinder und Jugendliche traumapädagogische Angebote brauchen. Früh und komplex traumatisierte Menschen, vor allem Kinder und Jugendliche, haben in der Vergangenheit zu wenig Beachtung erfahren (vgl. BMFSFJ 2009: 238). Gesellschaftliche Veränderungen, steigende Anforderungen an Professionalisierung der pädagogisch Handelnden (vgl. Zimmermann 2016: 12 ff.), ausgebauter Kinderschutz, Inklusionsförderung an den Schulen, viele geflüchtete Jungen und Mädchen, die momentan die Jugendhilfe, das Gesundheits-, aber auch das Bildungssystem beschäftigen, überfordern oft die pädagogischen Fachkräfte und Institutionen, die auf Abhilfe hoffen.

Traumapädagogik scheint zunächst eine Lösung anzubieten. Sie trifft momentan auf eine Zeit, die dankend das Angebot annimmt, ohne es zu hinterfragen. Viele Praktiker aus verschiedenen Arbeitsfeldern melden Interesse an psychotraumatologischem Wissen an. Viele Weiterbildungs- und Fortbildungsinstitute sind entstanden, die momentan stark nachgefragt werden (vgl. Schirmer 2016: 439). „Traumaarbeit ist ein Produkt, das verkauft wird, und der Gewinn hängt nur davon ab, wie viel verkauft wird, und nicht davon, wie gut das Produkt ist“ (Becker 2006: 207).

Die Arbeit mit traumatisierten Kindern und Jugendlichen findet an der Schnittstelle zwischen verschiedenen Hilfesystemen (Jugendhilfe, Gesundheitswesen, Schulwesen u. a.) statt und fällt unter Zuständigkeiten des Bundes, der Länder und Kommunen (vgl. Fergert u. a. 2013: 307). Die Unklarheit über Zuständigkeiten erschwert es, Verantwortung zu übernehmen. Oft müsse die Jugendhilfe das betroffene Kind übernehmen, da es an geeigneten Therapieplätzen mangele (vgl. Ferget u. a. 2013: 302, vgl. Hensel 2013: 84). Obwohl einige traumatherapeutische Verfahren existieren, die auch bei Kindern und Jugendlichen ansetzbar sind, ist adäquate Versorgung dieser noch lange nicht gewährleistet. Vor allem die psychotherapeutische Versorgung der komplex traumatisierten Kinder und Jugendlichen, die zusätzlich an begleitenden Störungen leiden, bedarf noch weiterer Anstrengungen, da diese andere Zugänge brauchen als nach Monotraumata, so Landolt und Hensel (2008: 305). Traumatherapie im Kindes- und Jugendalter gehört nicht zur Grundausbildung von Kinder- und JugendlichenpsychotherapeutInnen (ebd.: 307). In der beruflichen Praxis der Fachkräfte fehlen gelingende Kooperationsmodelle zwischen den Professionellen und den Beteiligten bzw. den zu beteiligenden Hilfesystemen (vgl. Gahleitner \& Homfeld 2016: 322). Eine gelungene, den KlientInnen dienliche Kooperation erfordert entsprechende Strukturen und Methoden sowie die Bereitstellung angemessener Ressourcen (vgl. ebd.). Strauß (2016: 456) nennt das „neue Fach“ Traumapädagogik einen „diskursiven Umweg“ in der Jugendhilfe, der die fällige politische Auseinandersetzung aussetze, und schlägt vor, von \textit{traumasensiblen} Ansätzen zu sprechen. Diese würden mehr Platz für die Beschreibung der notwendigen Schnittstellen zwischen z. B. der Jugendhilfe und dem Gesundheitssystem lassen (vgl. ebd.).

Der Traumapädagogik mangelt es an einer explizit pädagogischen Perspektive auf Trauma, so Zimmerman (2016: 47). Die in der Literatur der Traumapädagogik vorfindlichen Definitionsversuche der Traumata beziehen sich auf Neurowissenschaften, Psychiatrie, Bindungstheorie und bilden hauptsächlich die klinischen Kategorien ab. Traumap{\"a}dagogik leidet laut Zimmerman (2016: 47 ff.) unter einer gering ausgeprägten Theoriebildung. Die Frage nach dem Verständnis der Traumatisierung als pädagogische Kategorie bleibt offen. Fachkräfte aus Medizin und Therapie würden oft ihre Forschungsergebnisse und Strategien als pädagogisch relevant verstehen und auf die Bedeutung der Pädagogik für die Stabilisierung verweisen. Indes werde nicht deutlich, wie sich pädagogische Arbeit von der therapeutischen unterscheide. Die Haltungs- und Handlungsleitfäden der therapeutischen Stabilisierungstechniken in das pädagogische Setting direkt zu übernehmen, das ein anderes als das therapeutische ist, führt dazu, die Beziehungsorientierung zugunsten der Methode zu verlassen, die selten ohne Weiteres auf ein gruppenbezogenes pädagogisches Setting übertragbar ist (vgl. ebd.). Traumapädagogik hilft, die neuesten Traumata-Erkenntnisse zu verbreiten, die wiederum neue Zugänge für das Verstehen des Kindes/Jugendlichen und ihrer/seiner Schwierigkeiten eröffnen können (vgl. BAG-TP 2011: 4).  
Das Innovative an der Traumap{\"a}dagogik ist nach Schmid und Lang (2012: 342) vor allem , dass das pädagogische Konzept aus den aktuellen wissenschaftlichen Erkenntnissen der Psychotraumatologie begründet wird, die pädagogischen Fachkräfte in die Konzeptentwicklung einbezogen werden und das Konzept konsequent praktisch umgesetzt wird, indem die Strukturen der Einrichtung verändert werden. Die dem Entwicklungsstand angemessene Psychoedukation des Kindes oder der/des Jugendlichen und die schulübergreifende Offenheit in der Auswahl der Methoden, solange sie sich aus der Psychotraumatologie begründen lassen, seien weitere Vorteile der Traumapädagogik (vgl. ebd.: 348). Eine große Stärke der traumapädagogischen Konzepte sei es des Weiteren, die internen Strukturen der Einrichtung durch die Formulierung von Anforderungen an Räumlichkeiten, Abläufe und Leitungsstrukturen konsequent einzubeziehen. Die volle Realisierung dieser Konzepte in der stationären Jugendhilfe ist laut Schmid und Lang (2012: 348) aufgrund der verfügbaren Ressourcen eher in Form der Intensivgruppen möglich. Mit der Traumapädagogik lassen sich alltägliche pädagogische Leistungen legitimieren sowie Unterstützungssysteme und Strukturen ableiten (vgl. ebd.). Der Bedarf an Ressourcen und Rahmenbedingungen sei besser begründbar, wenn dieser aus der psychotraumatologischen Perspektive heraus geschehe, so Schmid u. a. (2014: 183). Dabei sei nicht die festgestellte Traumatisierung, sondern der p{\"a}dagogische Bedarf eines Kindes entscheidend (vgl. Schmid \& Lang 2012: 349). Die Traumapädagogik (bzw. die Verbindung zur Traumatologie) lasse sowohl die pädagogischen Fachkräfte als auch Kinder und Jugendliche als förderungs- und schutzwürdiger erscheinen (vgl. ebd).

\begin{quote}
\small{„Der Begriff der Traumatisierung ist somit nicht zwingend notwendig, scheint aber den Wechsel der Perspektive von Problemkindern und potenziellen T{\"a}tern zu Opfern mit einem Hilfebedarf zu beg{\"u}nstigen und den direkten Link zur Psychotraumatologie mit ihren neurobiologischen Aspekten zu erleichtern“ (Schmid \& Lang 2012: 349).}
\end{quote}

Dennoch sollte gefragt werden, welchen Einfluss es auf junge Menschen hat, als „Trauma-Opfer“ tituliert zu werden (vgl. Strauß 2016: 454). Eine Pädagogik, die die Subjektorientierung propagiere, müsse fragen, ob die betroffenen Kinder und Jugendliche mit dem neuen Deutungsmuster: „Opfer mit Hilfebedarf“ einverstanden seien.

\subsection{Zur Attraktivität des Traumabegriffes: \textit{Traumap{\"a}dagogik} oder \textit{Traumasensibilität}?}
Der Begriff \textit{Traumapädagogik} wirkt verwirrend und wirft Fragen auf (vgl. Strauß 2016: 449). Was bedeutet das eigentlich? Ist es eine Pädagogik, die mit Traumata und traumatisierten Menschen umzugehen weiß? Bei solchen Bezeichnungen wie \textit{Musikpädagogik} oder \textit{Kunstpädagogik} erschließen sich wenigstens in Umrissen Basis und Methode der pädagogischen Arbeit. Das Verursachen von Traumatisierungen sei nicht das Ziel traumapädagogischer Konzepte. Begrifflich suggeriere der Name der „neuen Fachdisziplin“ das Gegenteil der Intention, so Strauß (2016: 449). Strauß (ebd.: 451 ff.) kritisiert das Benutzen eines Begriffes aus dem klinisch-medizinischen Bereich als ein „Label“ für die „neue Pädagogik“ und fragt, wozu die Pädagogik diese Etikettierung brauche und was sie damit anrichte (vgl. ebd.). Er sieht das Koppeln der Haltungen, Strategien und entsprechender Infrastruktur auf scheinbar objektivierbare klinische Problemstellungen (hier: Traumatisierung) als typische Strategie, die in Deutschland unter anderem geschichtliche Hintergründe habe (vgl. Strauß 2016: 451). Die notwendigen Veränderungen würden nur für wenige, ausgewählte Gruppen in Verbindung mit der Erwartung, die Symptome der Betroffenen zu beseitigen und damit Hilfebedarf abzustellen, realisiert. Damit werde „die Entwertung der Hilfebedürftigen […] ökonomisch legitimiert“ (ebd.: 415). Anstatt nach allgemeinen Ursachen/strukturellen Problemen zu fragen, werde mit einer auf ein klinisches Konzept zugreifenden „neuen“ Pädagogik eine Gruppe definiert und gesondert behandelt.

\begin{quote}
\small{„Trauma und Gewalt als gesamtgesellschaftliche Probleme anzugehen, erfordert selbstreflektive Akteure und Einrichtungen. Die Ausgliederung des Themas in privatwirtschaftlich organisierte Weiterbildungen verstärkt die Illusion eines funktionierenden Hilfesystems. Mehr noch: ein Helfersystem, welches den eigenen Beschäftigten und den Hilfebedürftigen einen Opferstatus annötigt, hat jeden emanzipatorischen Anspruch aufgegeben“ (Strauß 2016: 455 f.).}
\end{quote}

\textit{Traumasensible} Ansätze könnten dazu beitragen, dass eine politische und gesamtgesellschaftliche Auseinandersetzung mit Trauma und Gewalt, aber auch mit den strukturellen Gegebenheiten des pädagogischen Handelns stattfindet (vgl. ebd.). Kühner (2005: 165 ff.) fragt nach der Attraktivität des Traumabegriffs und den Ambivalenzen, die dieser hervorrufe. Mit tiefem Leid und Schmerz konfrontiert zu sein, sei eine Herausforderung sowohl für die direkt Betroffenen als auch für die Helfenden. Die Deklaration als Trauma sei eine Möglichkeit, über das „Schlimme“ zu sprechen, gebe Halt, Erklärung und Anerkennung. Zugleich schaffe der Begriff Distanz zum Geschehenen, mache es fremder und abstrakter, so Kühner. Hinter dem Begriff \textit{Trauma} verbergen sich auch bestimmte Konzepte, die handlungsfähiger machen sollen und vom aktuellen Traumadiskurs sowie seinem Kontext abhängen (vgl. ebd.: 167). Umso wichtiger ist es, sich „immer wieder vom Begriff Trauma und damit von der Illusion zu distanzieren, man wüsste schon, was das ist“ (Kühner 2005: 171) und wie es zu behandeln sei.

Die inflationäre und vereinfachte Verwendung des medizinisch orientierten Traumakonzepts (vgl. Becker 2006) verführt dazu, statt die bestehenden Erkenntnisse und Handlungsstrategien (z. B. zu sexueller Gewalt oder zum Kinderschutz) zu ergänzen, durch „das Neue“ (hier: Traumapädagogik) zu ersetzen (vgl. BMFSFJ 2009: 238). Der 13. Kinder- und Jugendbericht der Bundesregierung hat traumatisierte Kinder und Jugendliche in den Blick genommen und mehr Traumasensibilität gefordert, aber auch, die Erkenntnisse über Traumata dazu zu nutzen, Kinder und Jugendliche zu unterstützen (vgl. BMFSFJ 2009: 239).

\begin{quote}
\small{„Traumasensible Fachkräfte sollten jedoch keine Diagnosen stellen, sondern vielmehr Reflexionsräume schaffen, in denen Vermutungen ausgetauscht und reflektiert werden können und im gemeinsamen Abwägen entschieden wird, ob ein Kind von Fachleuten diagnostiziert und gegebenenfalls behandelt werden sollte“ (BMFSFJ 2009, S. 239).}
\end{quote}

Zugleich bedeutet Traumasensibilität den reflektierenden Umgang mit Ungewissheit und das Aushalten dessen. Gewarnt wird davor, nach vereinfachten kausalen Zusammenhängen zu suchen (vgl. ebd.: 240). Zur Traumasensibilität gehöre auch, die Begrenztheit der eigenen Möglichkeiten anzuerkennen, einem traumatisierten Kind zu helfen. Die Fachpraxis stoße an ihre Grenzen, die wiederum den Mangel an interdisziplinären und interprofessionellen Angeboten aufzuzeigen hätten. In diesen Angeboten müssten sich stabile pädagogische und therapeutische Unterstützung gegenseitig ergänzen. Weiterbildung und Beratungsbedarf der Fachkräfte wird ebenso thematisiert (vgl. ebd.).
