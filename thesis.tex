% !TeX encoding = UTF-8

%===============================================================================
% Font options are:
%   plain (default), serif (uses Palladio), sans-serif (uses Paratype Sans)
% Layout options are:
%   article (default, no chapters), book (for longer texts, offers \chapter)
% Paragraph options are:
%   noparskip (default, no spacing between paragraphs), parskip (spaced)
\documentclass[serif,article,noparskip]{agse-thesis}

% Global parameters, replace with actual values.
\newcommand{\thesisTitle}{Traumapädagogik - wann und wozu und wann nicht mehr?}
\newcommand{\thesisSubtitle}{Zur Attraktivität des \textit{Traumas} für die Pädagogik}
% -> You may use \par (but not \\) to format the title. If you do so, you'll
%    need to manually set the 'pdftitle' attribute below.
\newcommand{\studentName}{Olga Urbaniak}
%===============================================================================

\hypersetup{pdftitle={\thesisTitle}}
\hypersetup{pdfauthor={\studentName}}

% Blind texts, for demonstration only, not part of the actual template
\usepackage{lipsum}

\begin{document}

\coverpage[
    student/id=1234567,
    student/mail=olga.urbaniak@fu-berlin.de,
    thesis/type=Bachelorarbeit,            % optional, default: Bachelorarbeit
    thesis/group={Arbeitsgruppe Software Engineering},
                                           % optional, default: AGSE
    thesis/firstAdvisor={1. Gutachterin: Prof. Dr. Ulrike Urban-Stahl},            thesis/secondAdvisor={2. Gutachterin: Prof. Dr. Katrin Kaufmann},           % optional
    thesis/examiner={Prof. Dr. Mia Maus},
    thesis/examiner/2={Prof. Dr. Bob Bär}, % optional
    thesis/date=\today,                    % optional, default: \today
   %title/size=\LARGE,      % set this value to overwrite automatic font size
   %abstract/separate       % toggle this to move the abstract to its own page
]
{ % Your abstract here:
    \lipsum[1]
}

% !TeX encoding = UTF-8
\pagestyle{fancy}
\fancyhf{} %alle Kopf- und Fußzeilenfelder bereinigen
\fancyfoot[C]{\small{Fachbereich Erziehungswissenschaft und Psychologie
Prüfungsbüro Bachelorstudiengang Psychologie – Habelschwerdter Alle 45, 14195 Berlin}}
Name: Urbaniak \hfill Vorname: Olga \hfill Matrikelnr.: xxxx
\section*{Eidesstattliche Erklärung zur Bachelorarbeit}
Ich versichere, die Bachelorarbeit selbstständig und lediglich unter Benutzung der angegebenen Quellen und Hilfsmittel verfasst zu haben.
\\\\
Ich erkläre weiterhin, dass die vorliegende Arbeit noch nicht im Rahmen eines anderen Prüfungsverfahrens eingereicht wurde.
\\\\
Berlin, den \thesisDate
\clearpage
\pagestyle{fancy}
\fancyhf{} %alle Kopf- und Fußzeilenfelder bereinigen


\cleardoublepage

\tableofcontents

\cleardoublepage

\mainmatter

\input{1_introduction}
\input{2_fundamentals}
\input{3_main}
\input{4_conclusion}

\bibliography{bibliography}

\appendix
\include{5_appendix}

\end{document}
