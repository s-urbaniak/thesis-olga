% !TeX encoding = UTF-8

%===============================================================================
% Font options are:
%   plain (default), serif (uses Palladio), sans-serif (uses Paratype Sans)
% Layout options are:
%   article (default, no chapters), book (for longer texts, offers \chapter)
% Paragraph options are:
%   noparskip (default, no spacing between paragraphs), parskip (spaced)
\documentclass[serif,article,noparskip]{agse-thesis}

% Global parameters, replace with actual values.
\newcommand{\thesisTitle}{Traumapädagogik - wann und wozu und wann nicht mehr?}
\newcommand{\thesisSubtitle}{Zur Attraktivität des \textit{Traumas} für die Pädagogik}
% -> You may use \par (but not \\) to format the title. If you do so, you'll
%    need to manually set the 'pdftitle' attribute below.
\newcommand{\studentName}{Olga Urbaniak}
%===============================================================================

\hypersetup{pdftitle={\thesisTitle}}
\hypersetup{pdfauthor={\studentName}}

% Blind texts, for demonstration only, not part of the actual template
\usepackage{lipsum}

\begin{document}

\coverpage[
    student/id=1234567,
    student/mail=olga.urbaniak@fu-berlin.de,
    thesis/type=Bachelorarbeit,            % optional, default: Bachelorarbeit
    thesis/group={Arbeitsgruppe Software Engineering},
                                           % optional, default: AGSE
    thesis/firstAdvisor={1. Gutachterin: Prof. Dr. Ulrike Urban-Stahl},            thesis/secondAdvisor={2. Gutachterin: Prof. Dr. Katrin Kaufmann},           % optional
    thesis/examiner={Prof. Dr. Mia Maus},
    thesis/examiner/2={Prof. Dr. Bob Bär}, % optional
    thesis/date=\today,                    % optional, default: \today
   %title/size=\LARGE,      % set this value to overwrite automatic font size
   %abstract/separate       % toggle this to move the abstract to its own page
]
{ % Your abstract here:
    \lipsum[1]
}

% !TeX encoding = UTF-8
\pagestyle{fancy}
\fancyhf{} %alle Kopf- und Fußzeilenfelder bereinigen


\cleardoublepage

\tableofcontents

\cleardoublepage

\mainmatter

% !TeX encoding = UTF-8
\section{Einführung}

Das Einführungskapitel beinhaltet ein paar praktische Hinweise zum Schreiben
der Abschlussarbeit, sowie eine Kurzdokumentation der bereitgestellten
\LaTeX-Klasse.
Die restlichen Kapitel dienen lediglich zu Demonstrationszwecken.

\subsection{Zur Abschlussarbeit als solche}

Neben der jeweiligen Studien- und Prüfungsordnung, die die förmlichen
Eigenschaften der Durchführung einer Abschlussarbeit regelt, sind folgende
Quellen hilfreich:
\begin{itemize}
    \item Studien- und Prüfungsordnungen der Informatikstudiengänge
    (\url{http://www.imp.fu-berlin.de/fbv/pruefungsbuero/Studien--und-Pruefungsordnungen/index.html})

    \item ThesisRules (\url{http://www.inf.fu-berlin.de/w/SE/ThesisRules}):

    Beschreibung des praktischen Ablaufs einer Abschlussarbeit in der AG
    Software Engineering von A bis Z.

    \item "`Technisches Schreiben"'
    (\url{http://www.mi.fu-berlin.de/wiki/pub/SE/SeminarRegeln/Technisches_Schreiben.pdf}):

    Ein von Lutz Prechelt verfasstes Dokument mit vielen praktischen
    Hinweisen zum Schreibteil (nicht nur) einer Abschlussarbeit.
\end{itemize}

\subsection{Zu dieser \LaTeX{}-Vorlage}

\subsubsection{Optionen der Dokumentenklasse}

Die Dokumentenklasse \texttt{agse-thesis} unterstützt verschiedene
Schriftarten:
\begin{lstlisting}[language={[LaTeX]TeX}]
% Standard LaTeX Schriftart
\documentclass[plain]{agse-thesis}

% Serifenschrift Palladino
\documentclass[serif]{agse-thesis}

% Serifenlose Schrift Paratype Sans
\documentclass[sans-serif]{agse-thesis}
\end{lstlisting}

Für kürzere Arbeiten, die mit Abschnitten (\texttt{\textbackslash{}section})
als oberste Gliederungsebene auskommen, reicht die Standard-Option
\texttt{article}.
Die Buch-Option \texttt{book} bietet darüber hinaus noch Kapitel
(\texttt{\textbackslash{}chapter}) an.
\begin{lstlisting}[language={[LaTeX]TeX}]
% Standard fuer kuerzere Arbeiten
\documentclass[article]{agse-thesis}

% Buch-Variante fuer umfangreiche Arbeiten mit vielen
% Gliederungselementen
\documentclass[book]{agse-thesis}
\end{lstlisting}

Ob zwischen den Absätzen im Text Abstände angezeigt werden sollen, oder ob
stattdessen die erste Zeile eines Absatzes eingerückt werden soll, kann mit
\texttt{parskip} bzw. \texttt{noparskip} eingestellt werden.
\begin{lstlisting}[language={[LaTeX]TeX}]
% Absaetze deutlich trennen
\documentclass[parskip]{agse-thesis}

% Absaetze nah bei einander, erste Zeile eingerueckt
\documentclass[noparskip]{agse-thesis}
\end{lstlisting}

Die Werte der drei o.g. Optionen können beliebig kombiniert werden:
\begin{lstlisting}[language={[LaTeX]TeX}]
% Einstellung des Beispieldokuments
\documentclass[serif,article,noparskip]{agse-thesis}
\end{lstlisting}

\subsubsection{Befehl \texttt{\textbackslash{}thesisTitle}}

Der Titel der Arbeit wird sowohl auf der Titelseite (siehe
\ref{sec:cmd-coverpage}) als auch für die PDF-Metainformationen benötigt.
Gesetzt wird der Titel durch das Definieren von
\texttt{\textbackslash{}thesisTitle}.

Für die Titelseite können manuell mit \texttt{\textbackslash{}par}
Zeilenumbrüche eingefügt werden um das Textbild zu verbessern (nicht hingegen
mit \texttt{\textbackslash\textbackslash}).
Sollte von dieser Möglichkeit Gebrauch gemacht werden, muss der Titel für die
PDF-Metainformationen manuell gesetzt werden
(\texttt{\textbackslash{}hypersetup\{pdftitle=\{...\}\}}).


\subsubsection{Befehl \texttt{\textbackslash{}coverpage}}
\label{sec:cmd-coverpage}

Die Titelseite der Abschlussarbeit wird mit dem
\texttt{\textbackslash{}coverpage}-Befehl erzeugt.
Dessen Ausgabe wird über eine Reihe von Schlüssel-Wert-Paaren konfiguriert
(siehe \autoref{tab:coverpage-config}).
\begin{table}[h]
\begin{center}
    \begin{tabular}{|l|L{5.5cm}|L{4cm}|}
        \hline
        \textbf{Schlüssel} & \textbf{Funktion} & \textbf{Default-Wert} \\
        \hline
        \texttt{student/id} & Matrikel-Nummer & -- \\
        \texttt{student/mail} & E-Mail-Adresse & -- \\
        \texttt{thesis/type} & Art der Abschlussarbeit & "`Bachelorarbeit"' \\
        \texttt{thesis/group} & Arbeitsgruppe in der die Arbeit geschrieben
        wurde & "`Arbeitsgruppe Software Engineering"' \\
        \texttt{thesis/advisor} & \emph{optional:} Betreuer der Abschlussarbeit
        & -- \\
        \texttt{thesis/examiner} & Erstgutachter der Arbeit & -- \\
        \texttt{thesis/examiner/2} & \emph{optional:} Zweitgutachter der Arbeit
        & -- \\
        \texttt{thesis/date} & \emph{optional:} Datum der Abgabe & aktuelles
        Datum\\
        \texttt{title/size} & \emph{optional:} \LaTeX-Schriftgröße für den
        Titel (\zb \texttt{\textbackslash{}LARGE}) & wird automatisch gesetzt \\
        \texttt{abstract/separate} & \emph{optional:} Schlüssel ohne Wert;
        falls gesetzt, wird der Abstract auf eine eigene Seite gesetzt und die
        Titelseite ist "`luftiger"' & -- \\
        \hline
    \end{tabular}
    \caption{Schlüssel-Wert-Konfiguration des
    \texttt{\textbackslash{}coverpage}-Kommandos.}
    \label{tab:coverpage-config}
\end{center}
\end{table}
Das einzige Argument des Kommandos ist der Abstract der Arbeit.
Ein minimaler Aufruf könnte so aussehen:
\begin{lstlisting}[language={[LaTeX]TeX}, morekeywords={coverpage}]
\coverpage[
    student/id=1234567,
    student/mail=email@inf.fu-berlin.de,
    thesis/type=Masterarbeit,
    thesis/examiner={Prof. Dr. Mia Maus}
]
{
    Prokrastination ist ein gut verstandenes Verhalten,
    das auch vor Abschlussarbeitern mit Informatik-Hintergrund
    nicht halt macht.
    % ...
}
\end{lstlisting}


\subsubsection{Verbesserungen der \LaTeX-Vorlage}

Diese \LaTeX-Vorlage soll den Einstieg in das Setzen der Abschlussarbeit
erleichtern.
Die Vorlage selbst wird in einem öffentlichen Git-Repository in der
GitLab-Instanz des Fachbereiches verwaltet, welches gerne als Grundlage für die
eigene Ausarbeitung geklont werden darf:
\begin{lstlisting}[language=bash]
git clone https://git.imp.fu-berlin.de/agse/thesis-template
\end{lstlisting}
Änderungsvorschläge in Form von Merge-Requests sind jederzeit willkommen.

% !TeX encoding = UTF-8
\section{Grundlagen}

\lipsum[8]

Siehe \zb \cite{Dje06, DjeOezSal07, CocWil00}.

\lipsum[9-11]


% !TeX encoding = UTF-8
\section{Hauptteil}

Der folgende Programmcode ist nicht repräsentativ für das Ergebnis einer
erfolgreichen Abschlussarbeit.

\begin{lstlisting}
public class Main {
    public static void main(String[] args) {
        System.out.println("Hello World");
    }
}
\end{lstlisting}

% !TeX encoding = UTF-8
\section{Zusammenfassung}

...


\bibliography{bibliography}

\appendix
% !TeX encoding = UTF-8
\section{Anhang}

Quellcode der \LaTeX-Klasse \texttt{agse-thesis}:\footnote{Es ist nicht üblich,
den gesamten produzierten Quellcode bei einer Abschlussarbeit in Textform
abzugeben.}

\lstinputlisting[
    language={[LaTeX]Tex},
    morekeywords={ProvidesClass, DeclareOption, PassOptionsToClass,
        ProcessOptions, CurrentOption, LoadClass, RequirePackage, ifthenelse,
        ifcsdef, equal, definecolor, lstset, pgfkeys},
    basicstyle=\footnotesize\ttfamily,
    numbers=left,
    numberstyle=\footnotesize\ttfamily,
    stepnumber=5,
    inputencoding=utf8,
    extendedchars=true,
   literate={ä}{{\"a}}1 {ü}{{\"u}}1,
]{agse-thesis.cls}


\end{document}
